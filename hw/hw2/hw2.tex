\documentclass[11pt]{article}
\usepackage{graphicx}
\usepackage{color}
\usepackage{comment}
\usepackage{multirow}
\usepackage{askmaps}
\usepackage{amssymb}
\usepackage{amsmath}

% Wrap long URLs with hyphens
\PassOptionsToPackage{hyphens}{url}\usepackage{hyperref}
\usepackage{pdftexcmds}
\usepackage{upquote}
\usepackage{textcomp}
\usepackage{minted}
\usepackage[listings]{tcolorbox}
\usepackage{enumerate}
\usepackage{enumitem}
\usepackage{mathtools}
\DeclarePairedDelimiter{\ceil}{\Big\lceil}{\Big\rceil}

\tcbset{
texexp/.style={colframe=black, colback=lightgray!15,
         coltitle=white,
         fonttitle=\small\sffamily\bfseries, fontupper=\small, fontlower=\small},
     example/.style 2 args={texexp,
title={Question \thetcbcounter: #1},label={#2}},
}

\newtcolorbox{texexp}[1]{texexp}
\newtcolorbox[auto counter]{texexptitled}[3][]{%
example={#2}{#3},#1}

\setlength{\topmargin}{-0.5in}
\setlength{\textheight}{9in}
\setlength{\oddsidemargin}{0in}
\setlength{\evensidemargin}{0in}
\setlength{\textwidth}{6.5in}

% Useful macros
\newenvironment{tightlist}
{\begin{itemize}
 \setlength{\parsep}{0pt}
 \setlength{\itemsep}{-2pt}}
{\end{itemize}}

\newenvironment{titledtightlist}[1]
{\noindent
 ~~\textbf{#1}
 \begin{itemize}
 \setlength{\parsep}{0pt}
 \setlength{\itemsep}{-2pt}}
{\end{itemize}}

% Change spacing before and after section headers
\makeatletter
\renewcommand{\section}
{\@startsection {section}{1}{0pt}
 {-2ex}
 {1ex}
 {\bfseries\Large}}
\makeatother

\makeatletter
\renewcommand{\subsection}
{\@startsection {subsection}{1}{0pt}
 {-1ex}
 {0.5ex}
 {\bfseries\normalsize}}
\makeatother

% Reduce likelihood of a single line at the top/bottom of page
\clubpenalty=2000
\widowpenalty=2000

% Other commands and parameters
\pagestyle{myheadings}
\setlength{\parindent}{0in}
\setlength{\parskip}{10pt}

% Commands for register format figures.
\newcommand{\instbit}[1]{\mbox{\scriptsize #1}}
\newcommand{\instbitrange}[2]{\instbit{#1} \hfill \instbit{#2}}

\graphicspath{{./figs/}}

\begin{document}
\def\PYZsq{\textquotesingle}


\title{EE219C HW2: SMT}
\author{Vighnesh Iyer}
\date{}
\maketitle

\section{Bit-Twiddling Hacks}
\begin{enumerate}[label=(\alph*)]
  \item {\color{blue}Are the functions \texttt{f1} and \texttt{f2} in Figure 1 equivalent?}

    \bigskip
    \noindent
    \begin{minipage}[c]{0.4\textwidth}
      \begin{minted}{c}
int f1(int x) {
  int v0;
  if (x > 0) v0 = x;
  else v0 = -x;
  return v0;
}
      \end{minted}
    \end{minipage}
    \hfill
    \begin{minipage}[c]{0.4\textwidth}
      \begin{minted}{c}
int f2(int x) {
  int v1, v2;
  v1 = x >> 31;
  v2 = x ^ v1;
  return (v2 - v1);
}
      \end{minted}
    \end{minipage}
    \bigskip

    \texttt{f1} is an absolute value function. \texttt{f2} is first isolating the sign bit in \texttt{v1} then performing a 2s complement inversion if the sign bit is 1. So these functions should be equal. I encoded this validity question using the Z3 Python API:

    \begin{minted}{python}
x, v0, v1, v2 = BitVecs('x v0 v1 v2', 32)
s = Solver()
s.add(v0 != v2 - v1, v0 == If(x > 0, x, -x), v1 == x >> 32, v2 == x ^ v1)
print(s.check())
print(s.sexpr())
    \end{minted}

    The equality between the return values of \texttt{f1} and \texttt{f2} was inverted to check for validity. The results were:
    \begin{minted}{text}
unsat
(declare-fun v0 () (_ BitVec 32))
(declare-fun x () (_ BitVec 32))
(declare-fun v2 () (_ BitVec 32))
(declare-fun v1 () (_ BitVec 32))
(assert (distinct v0 (bvsub v2 v1)))
(assert (= v0 (ite (bvsgt x #x00000000) x (bvneg x))))
(assert (= v1 (bvashr x #x00000020)))
(assert (= v2 (bvxor x v1)))
(model-add v0
           ()
           (_ BitVec 32)
           (bvmul x (ite (bvsle x #x00000000) #xffffffff #x00000001)))
(model-add v2 () (_ BitVec 32) (bvxor x v1))
    \end{minted}

    Showing that \texttt{f1} and \texttt{f2} are functionally equivalent.

  \item {\color{blue}Are the functions \texttt{f3} and \texttt{f4} in Figure 1 equivalent?}

    \bigskip
    \noindent
    \begin{minipage}[c]{0.4\textwidth}
      \begin{minted}{c}
int f3(int x, int y) {
  int v0;
  if (x >= y) v0 = x;
  else v0 = y;
  return v0;
}
      \end{minted}
    \end{minipage}
    \hfill
    \begin{minipage}[c]{0.4\textwidth}
      \begin{minted}{c}
int f4(int x, int y) {
  int v1, v2, v3;
  v1 = x ^ y;
  v2 = (-(x >= y));
  v3 = v1 & v2;
  return (v3 ^ y);
}
      \end{minted}
    \end{minipage}
    \bigskip

    \texttt{f3} is a max function. I used Z3 in the same manner:

    \begin{minted}{python}
x, y, v0, v1, v2, v3 = BitVecs('x y v0 v1 v2 v3', 32)
s = Solver()
s.add(v0 != v3 ^ y,
      v0 == If(x >= y, x, y),
      v1 == x ^ y,
      v2 == If(x >= y, BitVecVal(-1, 32), BitVecVal(0, 32)),
      v3 == v1 & v2
     )
print(s.check())
print(s.sexpr())
    \end{minted}

    These two functions are also equivalent:
    \begin{minted}{text}
unsat
(declare-fun v0 () (_ BitVec 32))
(declare-fun y () (_ BitVec 32))
(declare-fun x () (_ BitVec 32))
(declare-fun v1 () (_ BitVec 32))
(declare-fun v2 () (_ BitVec 32))
(declare-fun v3 () (_ BitVec 32))
(assert (distinct v0 (bvxor v3 y)))
(assert (= v0 (ite (bvsge x y) x y)))
(assert (= v1 (bvxor x y)))
(assert (= v2 (ite (bvsge x y) #xffffffff #x00000000)))
(assert (= v3 (bvand v1 v2)))
(model-add v0 () (_ BitVec 32) (ite (bvsle y x) x y))
(model-add v1 () (_ BitVec 32) (bvxor x y))
(model-add v2 () (_ BitVec 32) (ite (bvsle y x) #xffffffff #x00000000))
(model-add v3
           ()
           (_ BitVec 32)
           (let ((a!1 (bvor (bvnot (bvxor x y))
                            (bvnot (ite (bvsle y x) #xffffffff #x00000000)))))
             (bvnot a!1)))
    \end{minted}
\end{enumerate}
\end{document}
